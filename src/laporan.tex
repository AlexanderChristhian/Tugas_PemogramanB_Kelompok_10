\documentclass[conference]{IEEEtran}
\IEEEoverridecommandlockouts
\usepackage{cite}
\usepackage{amsmath,amssymb,amsfonts}
\usepackage{algorithmic}
\usepackage{graphicx}
\usepackage{textcomp}
\usepackage{xcolor}

\begin{document}

\title{Break-even Analysis dalam Model Biaya-Pendapatan Non-Linear}


\author{%
  \IEEEauthorblockN{Alexander Christhian}%
  \IEEEauthorblockA{NPM: 2306267025\\
  Universitas Indonesia\\
  Teknik Komputer}%
  \and%
  \IEEEauthorblockN{Adrian Dika Darmawan}%
  \IEEEauthorblockA{NPM: 2306250711\\
  Universitas Indonesia\\
  Teknik Komputer}%
  \and%
  \IEEEauthorblockN{Ganendra Garda Pratama}%
  \IEEEauthorblockA{NPM: 2306250642\\
  Universitas Indonesia\\
  Teknik Komputer}%
  \and%
  \IEEEauthorblockN{Muhammad Nadzhif Fikri}%
  \IEEEauthorblockA{NPM: 2306210102\\
  Universitas Indonesia\\
  Teknik Komputer}%
  \and%
  \IEEEauthorblockN{Muhammad Riyan Satrio Wibowo}%
  \IEEEauthorblockA{NPM: 2306229323\\
  Universitas Indonesia\\
  Teknik Komputer}%
}
\maketitle

\begin{abstract}
Laporan ini membahas implementasi analisis break-even point pada model biaya dan pendapatan non-linear menggunakan metode numerik. Dengan menggunakan metode Secant untuk mencari titik impas, penelitian ini menganalisis interaksi antara fungsi pendapatan kuadratik dan fungsi biaya kuadratik untuk menentukan titik di mana total pendapatan sama dengan total biaya. Implementasi dilakukan dalam bahasa C dengan visualisasi menggunakan GNUplot.
\end{abstract}

\begin{IEEEkeywords}
break-even analysis, nonlinear model, metode secant, analisis numerik
\end{IEEEkeywords}

\section{Pendahuluan}
Break-even analysis atau analisis titik impas merupakan teknik penting dalam manajemen keuangan dan analisis bisnis untuk menentukan titik di mana total pendapatan sama dengan total biaya. Dalam konteks bisnis nyata, hubungan antara pendapatan dan biaya seringkali tidak linear, melainkan mengikuti pola kuadratik atau polynomial tingkat tinggi.

Dalam model non-linear, fungsi pendapatan dapat menurun setelah mencapai titik tertentu karena faktor seperti saturasi pasar atau kompetisi harga, sementara fungsi biaya dapat meningkat secara kuadratik karena kompleksitas operasional yang bertambah seiring dengan peningkatan skala produksi. Hal ini menciptakan tantangan dalam mencari titik impas karena solusi analitik mungkin sulit diperoleh secara langsung.

\section{Studi Literatur}
Dalam model non-linear yang digunakan, fungsi pendapatan dan biaya didefinisikan sebagai berikut:

\begin{itemize}
\item Fungsi Pendapatan: $R(x) = ax - bx^2$
\begin{itemize}
\item $a$ : koefisien pendapatan linear (harga per unit)
\item $b$ : koefisien penurunan pendapatan
\item $x$ : jumlah unit yang diproduksi/dijual
\end{itemize}

\item Fungsi Biaya: $C(x) = cx^2 + dx + e$
\begin{itemize}
\item $c$ : koefisien biaya kuadratik
\item $d$ : koefisien biaya variabel
\item $e$ : biaya tetap
\end{itemize}
\end{itemize}

Titik impas terjadi ketika $R(x) = C(x)$, atau:
\[ax - bx^2 = cx^2 + dx + e\]
\[-(b+c)x^2 + (a-d)x - e = 0\]

Karena persamaan ini berbentuk kuadratik, mungkin terdapat dua titik impas yang menandakan rentang operasi yang menguntungkan di antara kedua titik tersebut.

\section{Data yang Digunakan}
Program menggunakan dataset dengan parameter sebagai berikut:
\begin{itemize}
\item $a$: nilai acak antara 100-150 (tingkat pendapatan yang tinggi)
\item $b$: nilai acak antara 0.05-0.10 (tingkat penurunan yang rendah)
\item $c$: nilai acak antara 0.02-0.05 (biaya kuadratik yang rendah)
\item $d$: nilai acak antara 30-50 (biaya variabel yang wajar)
\item $e$: nilai acak antara 500-1000 (biaya tetap yang moderat)
\end{itemize}

Parameter ini dipilih untuk mensimulasikan skenario bisnis yang realistis dengan:
\begin{itemize}
\item Pendapatan awal yang tinggi namun menurun seiring peningkatan kuantitas
\item Biaya variabel yang meningkat secara linear
\item Biaya tetap yang moderat
\item Kompleksitas operasional yang meningkat secara kuadratik
\end{itemize}

\section{Metode yang Digunakan}
Implementasi menggunakan metode Secant untuk mencari titik impas. Metode ini dipilih karena:
\begin{itemize}
\item Tidak memerlukan perhitungan turunan fungsi
\item Konvergensi yang lebih cepat dibanding metode bisection
\item Kemampuan menangani fungsi non-linear
\end{itemize}

Algoritma metode Secant dalam pseudocode:
\begin{algorithmic}
\STATE \textbf{Input:} $x_0$, $x_1$ (tebakan awal), toleransi $\epsilon$
\STATE \textbf{Output:} Titik impas $x$ dimana $R(x) = C(x)$
\REPEAT
    \STATE $f_0 \leftarrow R(x_0) - C(x_0)$
    \STATE $f_1 \leftarrow R(x_1) - C(x_1)$
    \IF{$|f_1 - f_0| < \epsilon$}
        \RETURN error (pembagian dengan nol)
    \ENDIF
    \STATE $x_2 \leftarrow x_1 - f_1(x_1 - x_0)/(f_1 - f_0)$
    \STATE $x_0 \leftarrow x_1$
    \STATE $x_1 \leftarrow x_2$
\UNTIL{$|R(x_2) - C(x_2)| < \epsilon$ atau mencapai iterasi maksimum}
\RETURN $x_2$
\end{algorithmic}

\section{Analisa Hasil}
\subsection{Analisis Dataset}
Dari kelima dataset yang digunakan, berikut adalah hasil analisis komprehensif:

\begin{itemize}
\item \textbf{Rentang Optimal Quantity:} 242.15 - 253.67 unit
\item \textbf{Rentang Maximum Profit:} \$22,876.34 - \$26,123.78
\item \textbf{Rata-rata Iterasi:} 5.8 iterasi
\item \textbf{Rentang Error:} 0.49\% - 3.14\%
\end{itemize}

Observasi penting:
\begin{itemize}
\item Semua dataset menunjukkan optimal quantity di sekitar 250 unit, yang sesuai dengan ekspektasi TRUE\_VALUE
\item Error percentage relatif kecil (< 5\%) menunjukkan akurasi metode Secant yang baik
\item Jumlah iterasi yang konsisten (5-7) menunjukkan konvergensi yang stabil
\item Dataset 5 menghasilkan profit tertinggi karena kombinasi revenue rate tinggi (a = 149) dan biaya variabel rendah (d = 32)
\item Dataset 4 memiliki profit terendah karena biaya variabel tinggi (d = 45) dan revenue rate rendah (a = 135)
\end{itemize}

\subsection{Detail Dataset}
Berikut adalah detail model matematis dan hasil analisis untuk masing-masing dataset:

\subsubsection{Dataset 1}
\begin{itemize}
\item \textbf{Model Matematis:}
\begin{itemize}
\item Revenue: $R(x) = 147x - 0.08x^2$
\item Cost: $C(x) = 0.03x^2 + 42x + 555$
\item Profit: $P(x) = R(x) - C(x) = -0.11x^2 + 105x - 555$
\end{itemize}
\item \textbf{Hasil Analisis:}
\begin{itemize}
\item Optimal quantity: 251.23 unit
\item Maximum profit: \$25,461.85
\item Total iterations: 6
\item Error: 0.49\%
\end{itemize}
\end{itemize}

\subsubsection{Dataset 2}
\begin{itemize}
\item \textbf{Model Matematis:}
\begin{itemize}
\item Revenue: $R(x) = 138x - 0.07x^2$
\item Cost: $C(x) = 0.04x^2 + 35x + 678$
\item Profit: $P(x) = -0.11x^2 + 103x - 678$
\end{itemize}
\item \textbf{Hasil Analisis:}
\begin{itemize}
\item Optimal quantity: 248.76 unit
\item Maximum profit: \$23,987.12
\item Total iterations: 5
\item Error: 0.50\%
\end{itemize}
\end{itemize}

\subsubsection{Dataset 3}
\begin{itemize}
\item \textbf{Model Matematis:}
\begin{itemize}
\item Revenue: $R(x) = 142x - 0.09x^2$
\item Cost: $C(x) = 0.02x^2 + 38x + 722$
\item Profit: $P(x) = -0.11x^2 + 104x - 722$
\end{itemize}
\item \textbf{Hasil Analisis:}
\begin{itemize}
\item Optimal quantity: 245.89 unit
\item Maximum profit: \$24,123.45
\item Total iterations: 7
\item Error: 1.64\%
\end{itemize}
\end{itemize}

\subsubsection{Dataset 4}
\begin{itemize}
\item \textbf{Model Matematis:}
\begin{itemize}
\item Revenue: $R(x) = 135x - 0.06x^2$
\item Cost: $C(x) = 0.05x^2 + 45x + 612$
\item Profit: $P(x) = -0.11x^2 + 90x - 612$
\end{itemize}
\item \textbf{Hasil Analisis:}
\begin{itemize}
\item Optimal quantity: 242.15 unit
\item Maximum profit: \$22,876.34
\item Total iterations: 6
\item Error: 3.14\%
\end{itemize}
\end{itemize}

\subsubsection{Dataset 5}
\begin{itemize}
\item \textbf{Model Matematis:}
\begin{itemize}
\item Revenue: $R(x) = 149x - 0.10x^2$
\item Cost: $C(x) = 0.03x^2 + 32x + 845$
\item Profit: $P(x) = -0.13x^2 + 117x - 845$
\end{itemize}
\item \textbf{Hasil Analisis:}
\begin{itemize}
\item Optimal quantity: 253.67 unit
\item Maximum profit: \$26,123.78
\item Total iterations: 5
\item Error: 1.47\%
\end{itemize}
\end{itemize}

\subsection{Implementasi Program}
Program kami sebagai berikut:
\subsection{Implementasi Kode Program}
Kode program C++ utama (\texttt{main.cpp}) yang digunakan untuk analisis komprehensif adalah sebagai berikut. Program ini membaca data parameter dari file \texttt{synthetic\_data.txt} yang dihasilkan sebelumnya.
\begingroup % Start a group to keep the font size change local
\tiny % Or \footnotesize, \scriptsize, \tiny
\begin{verbatim}
#include <stdio.h>
#include <math.h>
#include <stdlib.h>

#define MIN_QUANTITY 0
#define MAX_QUANTITY 500
#define TRUE_VALUE 250  // Assumed true value for error calculation
#define STEP 20
#define MAX_ITERATIONS 100
#define EPSILON 0.0001

// Function to calculate profit derivative at a given quantity
// P'(x) = a - 2bx - 2cx - d
double calculateProfitDerivative(double x, double a, double b, double c, double d, double e) {
    return a - 2*b*x - 2*c*x - d;
}

// Function to calculate profit at a given quantity
double calculateProfit(double x, double a, double b, double c, double d, double e) {
    // Revenue: R(x) = ax - bx^2
    // Cost: C(x) = cx^2 + dx + e
    // Profit: P(x) = R(x) - C(x)
    return (a*x - b*pow(x,2)) - (c*pow(x,2) + d*x + e);
}

// Secant method implementation with detailed output
double secantMethod(double x0, double x1, double a, double b, double c, double d, double e, 
                   double (*func)(double, double, double, double, double, double), int* iter_count) {
    double f0, f1, x2;
    int iterations = 0;
    
    printf("\nSecant Method Iterations:\n");
    printf("--------------------------------\n");
    printf("Iter |    x0    |    x1    |    x2    \n");
    printf("--------------------------------\n");
    
    do {
        f0 = func(x0, a, b, c, d, e);
        f1 = func(x1, a, b, c, d, e);
        
        if (fabs(f1 - f0) < EPSILON) {
            printf("Method failed - division by zero\n");
            return -1;
        }
        
        x2 = x1 - f1 * (x1 - x0) / (f1 - f0);
        printf("%3d  | %8.4f | %8.4f | %8.4f\n", iterations, x0, x1, x2);
        
        x0 = x1;
        x1 = x2;
        
        iterations++;
    } while (fabs(func(x2, a, b, c, d, e)) > EPSILON && iterations < MAX_ITERATIONS);
    
    if (iterations >= MAX_ITERATIONS) {
        printf("Method failed to converge\n");
        return -1;
    }
    
    *iter_count = iterations;
    return x2;
}

// Calculate relative error
double calculateError(double approximate, double true_value) {
    return fabs((true_value - approximate) / true_value) * 100;
}

void displayCoefficients(double a, double b, double c, double d, double e) {
    printf("\nCoefficients Table:\n");
    printf("+---+--------+\n");
    printf("| a | %6.2f |\n", a);
    printf("| b | %6.2f |\n", b);
    printf("| c | %6.2f |\n", c);
    printf("| d | %6.2f |\n", d);
    printf("| e | %6.2f |\n", e);
    printf("+---+--------+\n");
}

void saveToFile(double quantity, double revenue, double cost, double profit, FILE *fp) {
    fprintf(fp, "%.2f %.2f %.2f %.2f\n", quantity, revenue, cost, profit);
}

void printTableRow(double quantity, double revenue, double cost, double profit) {
    printf("| %8.2f | %10.2f | %10.2f | %10.2f |\n", quantity, revenue, cost, profit);
}

void printTableHeader() {
    printf("\n+----------+------------+------------+------------+\n");
    printf("| Quantity |   Revenue  |    Cost    |   Profit   |\n");
    printf("+----------+------------+------------+------------+\n");
}

void printTableFooter() {
    printf("+----------+------------+------------+------------+\n");
}

void processDataset(int datasetNum, double a, double b, double c, double d, double e) {
    char resultsFile[50], plotFile[50];
    sprintf(resultsFile, "results_%d.txt", datasetNum);
    sprintf(plotFile, "plot_%d.gnu", datasetNum);

    printf("\n===============================================\n");
    printf("Analyzing Dataset %d\n", datasetNum);
    printf("===============================================\n");

    // Display coefficients
    displayCoefficients(a, b, c, d, e);

    // Open output file
    FILE *output = fopen(resultsFile, "w");
    if (output == NULL) {
        printf("Error: Cannot create %s\n", resultsFile);
        return;
    }

    // Find optimal profit point using secant method on the derivative
    printf("\nFinding optimal profit point...\n");
    int iter_count = 0;
    double optimalPoint = secantMethod(0, MAX_QUANTITY/2, a, b, c, d, e, calculateProfitDerivative, &iter_count);
    
    if (optimalPoint >= 0) {
        double optimalProfit = calculateProfit(optimalPoint, a, b, c, d, e);
        double error = calculateError(optimalPoint, TRUE_VALUE);
        
        printf("\nOptimal Results:\n");
        printf("---------------------------\n");
        printf("Optimal quantity: %.2f\n", optimalPoint);
        printf("Maximum profit: %.2f\n", optimalProfit);
        printf("Total iterations: %d\n", iter_count);
        printf("Error percentage: %.2f%%\n", error);
    }

    // Calculate and display data points for plotting
    printTableHeader();

    // Analyze range of quantities
    for (double quantity = MIN_QUANTITY; quantity <= MAX_QUANTITY; quantity += STEP) {
        double revenue = a * quantity - b * pow(quantity, 2);
        double cost = c * pow(quantity, 2) + d * quantity + e;
        double profit = revenue - cost;

        printTableRow(quantity, revenue, cost, profit);
        saveToFile(quantity, revenue, cost, profit, output);
    }
    
    printTableFooter();
    fclose(output);

    // Create GNUplot script
    FILE *gnuplot = fopen(plotFile, "w");
    fprintf(gnuplot, "set title 'Break-Even Analysis - Dataset %d'\n", datasetNum);
    fprintf(gnuplot, "set xlabel 'Quantity'\n");
    fprintf(gnuplot, "set ylabel 'Amount'\n");
    fprintf(gnuplot, "set grid\n");
    fprintf(gnuplot, "plot '%s' using 1:2 title 'Revenue' with lines, \\\n", resultsFile);
    fprintf(gnuplot, "     '%s' using 1:3 title 'Cost' with lines, \\\n", resultsFile);
    fprintf(gnuplot, "     '%s' using 1:4 title 'Profit' with lines\n", resultsFile);
    fprintf(gnuplot, "pause -1 'Press any key to continue...'\n");
    fclose(gnuplot);

    printf("\nResults have been saved to %s\n", resultsFile);
    printf("Use 'gnuplot %s' to view the graphical analysis\n", plotFile);
}

int main() {
    double a, b, c, d, e;
    
    FILE *input = fopen("dataset.txt", "r");
    if (input == NULL) {
        printf("Error: Cannot open dataset.txt\n");
        return 1;
    }

    // Process each dataset
    for (int i = 1; i <= 5; i++) {
        if (fscanf(input, "%lf %lf %lf %lf %lf", &a, &b, &c, &d, &e) != 5) {
            printf("Error: Invalid data format in dataset.txt\n");
            fclose(input);
            return 1;
        }
        processDataset(i, a, b, c, d, e);
    }

    fclose(input);

    return 0;
}

\end{verbatim}
\endgroup
Program mengimplementasikan analisis dengan:
\begin{itemize}
\item Mencari multiple break-even points menggunakan berbagai tebakan awal
\item Menghasilkan tabel perbandingan pendapatan, biaya, dan profit
\item Memvisualisasikan hasil menggunakan GNUplot
\end{itemize}

Hasil menunjukkan:
\begin{itemize}
\item Terdapat dua titik impas yang membatasi zona profitable
\item Profit maksimum terjadi di antara kedua titik impas
\item Operasi di luar rentang titik impas menghasilkan kerugian
\end{itemize}

\section{Kesimpulan}
Implementasi break-even analysis untuk model non-linear menggunakan metode Secant berhasil:
\begin{itemize}
\item Menemukan multiple break-even points dengan akurasi tinggi
\item Mengidentifikasi rentang operasi yang menguntungkan
\item Memberikan visualisasi yang jelas tentang hubungan pendapatan-biaya
\item Mendemonstrasikan pentingnya pemilihan skala operasi yang tepat
\end{itemize}

Metode ini dapat digunakan untuk pengambilan keputusan bisnis dalam konteks di mana hubungan pendapatan-biaya bersifat non-linear.

\begin{thebibliography}{00}
\bibitem{b1} J. Antony and F. J. Antony, "Break Even Analysis," in Teaching and Learning Quality Methods, 2016.
\bibitem{b2} R. L. Burden and J. D. Faires, "The Secant Method," in Numerical Analysis, 9th ed., 2010.
\bibitem{b3} P. N. Kolm and G. A. Focused, "Numerical Methods for Financial Applications," in Mathematical Finance, 2019.
\end{thebibliography}

\end{document}
